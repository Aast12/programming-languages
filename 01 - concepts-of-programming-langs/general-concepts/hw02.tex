\documentclass[]{article}
\usepackage[margin={1in}]{geometry}
\usepackage[spanish]{babel}
\usepackage{datetime}
\usepackage{amsmath}
\usepackage{listings}
\usepackage{verbatim}
\usepackage{fancyhdr}
\usepackage[obeyspaces]{url}
\usepackage{graphicx}
\graphicspath{ {.} }
\newdateformat{hwdate}{\THEDAY~de \monthname[\THEMONTH], \THEYEAR}
\newcommand{\class}{TC2006. Programming languages}
\newcommand{\hwname}{Homework 02}
\pagestyle{fancy}
\fancyhf{}
\lhead{\class \\ \hwname}
\rhead{Andrés Alam Sánchez Torres (A00824854) \\ \hwdate\today}
\lstset{breaklines=true,
        language=haskell,
        showstringspaces=false} 

\begin{document}
    \setlength{\headheight}{23.10004pt}
    \addtolength{\topmargin}{-11.10004pt}    

    \section{Classifying programming languages}
    
    \noindent
    Please select three programming languages from any source you prefer. For each of these programming languages, 
    provide the following information:

    \begin{itemize}
    
        \item A web page where the program is described (this can be the vendor’s web page).
        \item Provide a small example of the code. It can be the “Hello world!” example that is commonly
        used. There is no need to include functional code, just a piece that allows you to show the
        properties of the language.
        \item Classify the language based on how it expresses the solution (declarative or imperative) and briefly
        justify your answer.
        \item  Classify the language based on its paradigm (imperative, declarative, structured, object oriented,
        functional or logic) and briefly justify your answer.
        
    \end{itemize}

    \subsection{Haskell}

    \begin{itemize}
    
        \item Web page: \url{https://www.haskell.org/}
        \item Example code:
        \begin{lstlisting}
            sayHi name = putStrLn ("hello " ++ name)

            main = sayHi "Alam"
        \end{lstlisting}

        \item Classify the language based on how it expresses the solution: 
            Declarative. A program made with haskell is meant to be made out of only evaluation of mathematical functions, and not with a complex execution flow. 
            Despite it can be used as an imperative language, Haskell was meant to be used as a declarative language, and has a lot of features that forces the programmer 
            to do so (e. g. immutable variables, no if statements). 
        \item  Classify the language based on its paradigm:
            Functional. For the same reason specified above, most haskell programs are structured by only function evaluations.
        
    \end{itemize}

    \subsection{Crystal}

    \begin{itemize}
    
        \item Web page: \url{https://crystal-lang.org/}
        \item Example code:
        \begin{lstlisting}
            def sayHi(name) 
                puts "hello #{name}"
            end

            sayHi("Alam")
        \end{lstlisting}

        \item Classify the language based on how it expresses the solution: 
            Imperative. Crystal supports multiple statements that allow the modification of the program's flow, and allows mutation 
            of variables to create a state. This leads to the definition of programs made out of a set of instructions to be followed. 
        \item  Classify the language based on its paradigm:
            Is both structured and object oriented. It supports classes and but it is not tied to the use of them, a program can be made
            by only using primitive data types.
        
    \end{itemize}

    \subsection{Ocaml}

    \begin{itemize}
    
        \item Web page: \url{https://ocaml.org/}
        \item Example code:
        \begin{lstlisting}
            let sayHi name = Printf.printf "Hello %s" name;;
    
            sayHi "Alam"
        \end{lstlisting}

        \item Classify the language based on how it expresses the solution: 
            It's imilar to haskell, so it can be considered declarative as it is meant to be used by using function evaluations. 
            However, it still has imperative features.
        \item  Classify the language based on its paradigm:
            Functional and Object oriented. OCaml is a functional programming language, having no restriction in the use of functions, but it also is complemented
            by object oriented capabilities.
        
        
    \end{itemize}

\end{document}